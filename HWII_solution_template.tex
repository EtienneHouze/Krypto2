\documentclass[11pt,twoside]{article}
\usepackage[utf8]{inputenc}
\usepackage[T1]{fontenc}
\usepackage[english]{babel}
\usepackage{amsmath}
\usepackage{amscd}
\usepackage{amssymb}
\usepackage{multirow}
\usepackage{tabularx}
\usepackage{url}
\usepackage{fancyhdr}
\usepackage{lastpage}
\usepackage[a4paper,margin=2.5cm,hmarginratio=1:1]{geometry}

%%%%%%%%%%%%%%%%%%%%%%%%%%%%%%%%%%%%%%%%%%%%%%%%%%%%%%%%%%%%%%%%%%%%%%%%%%%
%%%%%%%%%%%%%% ENTER YOUR PERSONAL INFORMATION HERE %%%%%%%%%%%%%%%%%%%%%%%
%%%%%%%%%%%%%%%%%%%%%%%%%%%%%%%%%%%%%%%%%%%%%%%%%%%%%%%%%%%%%%%%%%%%%%%%%%%

% The info of your group members. Set Y and Z empty if you are not
% three.

% THIS FIRST ONE IS YOU.
\newcommand{\persnrX}{941212-T437}
\newcommand{\nameX}{\'Etienne}
\newcommand{\familynameX}{Houz\'e}
\newcommand{\emailX}{houze@kth.se}

% THE OTHER TWO ARE PEOPLE YOU HAVE DISCUSSED INFORMALLY WITH.
\newcommand{\persnrY}{123456-7890}
\newcommand{\nameY}{Mohammad}
\newcommand{\familynameY}{Al-Khwarizmi}
\newcommand{\emailY}{mohammad@alkhwarizmi.hi}

\newcommand{\persnrZ}{123456-7890}
\newcommand{\nameZ}{Ada}
\newcommand{\familynameZ}{Lovelace}
\newcommand{\emailZ}{ada@lovelace.hi}


%%%%%%%%%%%%%%%%%%%%%%%%%%%%%%%%%%%%%%%%%%%%%%%%%%%%%%%%%%%%%%%%%%%%%%%%%%%
%%%%%%%%%%%%% DO NOT TOUCH ANYTHING BELOW THIS LINE %%%%%%%%%%%%%%%%%%%%%%%
%%%%%%%%%%%%%%%%%%%%%%%%%%%%%%%%%%%%%%%%%%%%%%%%%%%%%%%%%%%%%%%%%%%%%%%%%%%

\newcounter{problem}
\renewcommand\theproblem{\arabic{problem}}
\newenvironment{problem}{%
  \bigbreak
  \refstepcounter{problem}\noindent
  \llap{\textbf{\theproblem}\quad}\ignorespaces
}{%
  \par\if@nadaten@solutions\relax\else\filbreak\fi
}
\newenvironment{problem*}{%
  \bigbreak
  \refstepcounter{problem}\noindent
  \llap{\textbf{\theproblem}\quad}\ignorespaces
}{%
  \par
}
\newcounter{subproblem}[problem]
\renewcommand{\thesubproblem}{\arabic{problem}\alph{subproblem}}
\newenvironment{subproblem}{%
  \refstepcounter{subproblem}%
  \list{}{}%
  \item\leavevmode
  \llap{\hbox to \leftmargini{\textbf{\thesubproblem}\hfil}}%
  \ignorespaces
}{%
  \endlist\if@nadaten@solutions\relax\else\filbreak\fi
}
\newenvironment{subproblem*}{%
  \refstepcounter{subproblem}%
  \list{}{}%
  \item\leavevmode
  \llap{\hbox to \leftmargini{\textbf{\thesubproblem}\hfil}}%
  \ignorespaces
}{%
  \endlist
}

\newcommand{\homeworknr}{II}
\newcommand{\homework}{Homework~\homeworknr}
\newcommand{\coursenumber}{DD2448}
\newcommand{\coursename}{\coursenumber~Foundations of cryptography}
\newcommand{\coursenick}{krypto18}

\lhead[\familynameX~\familynameY~\familynameZ]{\coursename}
\chead{}
\rhead[\coursename]{\familynameX~\familynameY~\familynameZ}
\lfoot[\thepage~(\pageref{LastPage})]{}
\cfoot{}
\rfoot[]{\thepage~(\pageref{LastPage})}

\fancypagestyle{firststyle}
{
   \fancyhf{}
   \fancyfoot[R]{\thepage~(\pageref{LastPage})}
}

\renewcommand{\headrulewidth}{0pt}


%%%%%%%%%%%%%%%%%%%%%%%%%%%%%%%%%%%%%%%%%%%%%%%%%%%%%%%%%%%%%%%%%%%%%%%%%%%
%%% HERE YOU CAN ADD YOUR OWN MACROS AND ENVIRONMENTS IN THE PREAMBLE %%%%%
%%%%%%%%%%%%%%%%%%%%%%%%%%%%%%%%%%%%%%%%%%%%%%%%%%%%%%%%%%%%%%%%%%%%%%%%%%%

% Add your macros here.

\newcommand{\TPOINTS}[1]{(#1T)}
\newcommand{\IPOINTS}[1]{(#1I)}

\begin{document}

%%%%%%%%%%%%%%%%%%%%%%%%%%%%%%%%%%%%%%%%%%%%%%%%%%%%%%%%%%%%%%%%%%%%%%%%%%%
%%%%%%%%%%%% THE FOLLOWING GENERATES THE HEADER %%%%%%%%%%%%%%%%%%%%%%%%%%%
%%%%%%%%%%%% DO NOT TOUCH THIS %%%%%%%%%%%%%%%%%%%%%%%%%%%%%%%%%%%%%%%%%%%%
%%%%%%%%%%%%%%%%%%%%%%%%%%%%%%%%%%%%%%%%%%%%%%%%%%%%%%%%%%%%%%%%%%%%%%%%%%%

\thispagestyle{firststyle}

\noindent
\hspace{0.3cm}{\huge\textbf{\coursename}}

\noindent
\rule{\textwidth}{1pt}

\noindent
\begin{tabularx}{\textwidth}{X|lll}
  & \textbf{Persnr} & \textbf{Name} & \textbf{Email} \\
\cline{2-4}
&\\[-0.3cm]
  \multirow{2}{*}{\textbf{\huge\homework}} & {\small\textbf{\persnrX}} & {\small\textbf{\nameX}} & {\small\textbf{\emailX}} \\
  & & {\small\textbf{\familynameX}} & \\
\cline{2-4}
  \multirow{2}{*}{\textbf{\huge\coursenick}} & {\small\persnrY} & {\small\nameY} & {\small\emailY} \\
  & & {\small\familynameY} & \\
  & {\small\persnrZ} & {\small\nameZ} & {\small\emailZ} \\
  & & {\small\familynameZ} & \\
&\\
[-0.2cm]
\end{tabularx}

\vspace{0.2cm}
\noindent
\rule{\textwidth}{1pt}

\vspace{0.5cm}

\pagestyle{fancy}

%%%%%%%%%%%%%%%%%%%%%%%%%%%%%%%%%%%%%%%%%%%%%%%%%%%%%%%%%%%%%%%%%%%%%%%%%%%
%%%%%%%%%%%%%%%%%%%%% YOUR SOLUTIONS START HERE %%%%%%%%%%%%%%%%%%%%%%%%%%%
%%%%%%%%%%%%%%%%%%%%%%%%%%%%%%%%%%%%%%%%%%%%%%%%%%%%%%%%%%%%%%%%%%%%%%%%%%%
%%                                                                       %%
%%  Do NOT remove any problem-, or subproblem environments, or nominal   %%
%%  ponts below. If you can not solve a problem, then you MUST simply    %%
%%  leave the "NOT SOLVED" string intact. This ensures that the          %%
%%  numbering is correct and it simplifies grading, leaving more time    %%
%%  to prepare lectures and help students.                               %%
%%                                                                       %%
%%%%%%%%%%%%%%%%%%%%%%%%%%%%%%%%%%%%%%%%%%%%%%%%%%%%%%%%%%%%%%%%%%%%%%%%%%%

\begin{problem}
  \TPOINTS{2} SOLVED
  
  A proof by reduction is a means of proving that a cryptographic system is secrue. This proof is vastly inspired by the problem reduction often used in complexity theory. It conssists of finding a problem $\mathcal{P}$ which is proved to be hard to solve, and reduce the breaking of the cryptosystem to the resolution of $\mathcal{P}$. Reducing the cryptosystem to $\mathbb{P}$ means that solving the cryptosystem implies solving an instance of $\mathcal{P}$. Thus, if the cryptosystem is easy to solve, then $\mathcal{P}$ should also be easy to solve. Since we know that $\mathcal{P}$ is hard, then we have proved that the cryptosytem is at least as hard to solve.
\end{problem}

\begin{problem}
  \begin{subproblem}
    \TPOINTS{2} SOLVED
    
    The definition of a negligible fucntion is : "a function $\epsilon : \mathbb{N} \rightarrow \mathbb{R}$ is negligible if and only if for every integer $c$ there exists a rank $n_c$ such that $\forall n > n_c, \epsilon(n) < \frac{1}{n^c}$ ". This implies that any negligble function tends to zero as $n$ tends to infinity.
    
    Moreover, let $l$ be a polynomial function and let us call $d$ its degree ($d$ is finite). Let us prove that $l\times \epsilon$ is negligible.
    
    Let $c$ be an integer. Since $\epsilon$ is negligible, we know that there exists a rank $n_0$ such that for all $n > n_0$ we have $\epsilon(n) < n^{-c-d}$. Then we have for all $n > n_0$ : $ l(n) \times \epsilon(n) = Kn^{-c}$, where $K$ is greater than the sum of all coefficients of $l$. Then there exists a rank $n_1$ such that, for all $n > n_1$ we finally have $l(n) \ times \epsilon{n} < n^{-c}$. So by definition, $l\times \epsilon$ is negligible.
  \end{subproblem}
  \begin{subproblem}
    \TPOINTS{1} NOT SOLVED % We leave this place holder here for improved readability.
  \end{subproblem}
  \begin{subproblem}
    \TPOINTS{2} NOT SOLVED % We leave this place holder here for improved readability.
  \end{subproblem}
\end{problem}

\begin{problem}
  \begin{subproblem}
    \TPOINTS{1} NOT SOLVED % We leave this place holder here for improved readability.
  \end{subproblem}
  \begin{subproblem}
    \TPOINTS{1} NOT SOLVED % We leave this place holder here for improved readability.
  \end{subproblem}
  \begin{subproblem}
    \TPOINTS{1} NOT SOLVED % We leave this place holder here for improved readability.
  \end{subproblem}
  \begin{subproblem}
    \TPOINTS{1} NOT SOLVED % We leave this place holder here for improved readability.
  \end{subproblem}
  \begin{subproblem}
    \TPOINTS{2} NOT SOLVED % We leave this place holder here for improved readability.
  \end{subproblem}
  \begin{subproblem}
    \TPOINTS{2} NOT SOLVED % We leave this place holder here for improved readability.
  \end{subproblem}
\end{problem}

\begin{problem}
  \begin{subproblem}
    \TPOINTS{2} NOT SOLVED % We leave this place holder here for improved readability.
  \end{subproblem}
  \begin{subproblem}
    \TPOINTS{1} NOT SOLVED % We leave this place holder here for improved readability.
  \end{subproblem}
  \begin{subproblem}
    \TPOINTS{3} NOT SOLVED % We leave this place holder here for improved readability.
  \end{subproblem}
\end{problem}

\begin{problem}
  \begin{subproblem}
    \TPOINTS{7} NOT SOLVED % We leave this place holder here for improved readability.
  \end{subproblem}
  \begin{subproblem}
    \TPOINTS{3} NOT SOLVED % We leave this place holder here for improved readability.
  \end{subproblem}
\end{problem}

\begin{problem}
  \begin{subproblem}
    \TPOINTS{4} NOT SOLVED % We leave this place holder here for improved readability.
  \end{subproblem}
  \begin{subproblem}
    \TPOINTS{2} NOT SOLVED % We leave this place holder here for improved readability.
  \end{subproblem}
\end{problem}

\begin{problem}
  \IPOINTS{4} NOT SOLVED % We leave this place holder here for improved readability.
\end{problem}

\begin{problem}
  \IPOINTS{3} NOT SOLVED % We leave this place holder here for improved readability.
\end{problem}

\begin{problem}
  \IPOINTS{4} NOT SOLVED % We leave this place holder here for improved readability.
\end{problem}

\begin{problem}
  \IPOINTS{2} NOT SOLVED % We leave this place holder here for improved readability.
\end{problem}

\end{document}

%%% Local Variables:
%%% mode: latex
%%% TeX-master: t
%%% End:
